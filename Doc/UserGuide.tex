\documentclass[a4paper,11pt]{extarticle}
\usepackage[a4paper]{geometry}
\geometry{verbose,tmargin=2cm,bmargin=2cm,lmargin=2cm,rmargin=2cm}

\usepackage{fontspec}
\defaultfontfeatures{Ligatures=TeX}
%\setmainfont{Linux Libertine O}
\setmainfont{FreeSerif}
%\setmonofont{Fira Mono}
\setmonofont{FreeMono}

\setlength{\parindent}{0cm}

\usepackage{hyperref}
\usepackage{url}
\usepackage{xcolor}

\usepackage{minted}
%\newminted{julia}{breaklines,fontsize=\footnotesize}
\newminted{julia}{breaklines}

\begin{document}

\title{User Guide for {\ttfamily ffr-ElectronicStructure.jl}}
\author{Fadjar Fathurrahman}
\date{}
\maketitle

\tableofcontents

\section{Introduction}

\verb|ffr-ElectronicStructure.jl| is a collection of programs\footnote{or scripts}
for learning electronic structure calculations.

Design of the package

Introduction to electronic structure calculation based on DFT

\section{Short introduction to Julia programming languange}

Common program structure that is employed in {\tt ffr-ElectronicStructure.jl}.

I tried to make it simple to follow, although the design is clearly not considered
good practice in software engineering.

Tried to make it modular, but no serious attempts to make the generic code
this result in a lot of code repetition

Using include, instead of module

\begin{juliacode}
# file main.jl

include("func1.jl")
include("func2.jl")

function main()
  # do something here
end

# call main function
main()
\end{juliacode}

\end{document}
